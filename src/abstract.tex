I have developed a real-time continuous gesture recognition system capable of
dealing with two important problems that have previously been neglected:
(a) smoothly handling two different kinds of gestures: those
characterized by distinct paths and those characterized by distinct hand
poses; and (b) determining how and when the system should respond to gestures.
The novel approaches in this thesis include: a probabilistic recognition
framework based on a flattened hierarchical hidden Markov model (HHMM) that
unifies the recognition of path and pose gestures;
and a method of using information from the
hidden states in the HMM to identify different
gesture phases (the pre-stroke, the nucleus and the post-stroke
phases), allowing the system to respond appropriately to both gestures that
require a discrete response and those needing a continuous response.

The system is extensible: new gestures can be added by recording 3-6 repetitions
of the gesture; the system will train an HMM model for the gesture
and integrate it into the existing HMM, in a process that takes only a few minutes.
Our evaluation shows that even using only a small number of
training examples (e.g. 6), the system can achieve an average $F_1$ score of
0.805 for
two forms of gestures.

To evaluate the performance of my system I collected a new dataset
(YANG dataset) that includes both path and pose gestures, offering a
combination currently lacking in the community and providing
the challenge of recognizing different types of gestures mixed
together. I also developed a novel hybrid evaluation metric that is
more relevant to real-time interaction with different gesture flows.