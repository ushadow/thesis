\chapter{Online Recognition Evaluation}\label{sec:evaluation}
The previous sections reported evaluation results most pertinent
to the sections under discussion. This chapter presents results
for the overall online recognition performance.

\section{Evaluation Protocol}
I evaluate the online gesture recognition performance using the YANG
dataset and the hybrid performance
metrics proposed in Section~\ref{sec:metrics}. The evaluation is based on the
assumption that all the path gestures are
discrete flow gestures, and pose gestures are continuous flow
gestures.
The survey results shown earlier in Section~\ref{sec:preferences} show that it
is important to allow users to quickly define and train their own gestures. Hence,
I evaluate my system using user dependent training and testing. For each user
in the dataset, I use the first 2 sessions of recording (6 samples per gesture)
as training examples, and the last 2 sessions as testing examples. The first 2
sessions have ``Rest'' prompts which help to do automatic segmentation on the
training data. All the results reported are the average test results from 10
users.

Figure~\ref{fig:recog-result} shows a visualization of the recognition result on
a test sequence. The figure highlights the challenges in the test sequences:
there are 21 gestures in each continuous unsegmented sequence; 
gestures sometimes follow one another immediately.

\begin{figure}[thb]
\centering
\includegraphics[trim=43mm 15mm 42mm 10mm, clip,
width=\columnwidth]{figures/recog_result.png}
\caption{Comparison between recognition result using online inference
and ground truth.
The colors correspond to different gestures. For discrete flow gestures
(Swipe Left/Right, Circle, Horizontal Wave), one color segment with a fixed
length is shown at the time of response. For continuous flow gestures, the
recognized gesture is shown at each frame indicating frame-by-frame responses.}
\label{fig:recog-result}
\end{figure}

\section{Effect of the Number of Principal Components}
PCA reduces the dimensionality of the feature vectors, and hence, reduces the
computational complexity. I use cross-validation to find the
best number of principal components that accounts for the
variation in the data while keeping the dimension at the
minimum. Figure~\ref{fig:pca} shows how the recognition $F_1$ scores depends on
the number of principal components used for the HOG descriptor.
The best average score is obtained with 15 principal components.

\begin{figure}[tbh]
\centering
\includegraphics[width=\columnwidth]{figures/f1_pca.png}
\caption{Graph showing how $F_1$ scores for discrete flow gestures, continuous
flow gestures and the average scores change with the number of principal
components used for the HOG descriptor.}
\label{fig:pca}
\end{figure}

\section{Compare Different Topologies}
In this section, I compare my unified framework with a common HMM-based approach
used in previous works~\cite{sharma00, Starner95}, which use the same
topology for all gestures.

Table~\ref{tab:result} compares the results between the two methods.
The third column is the result from treating the two \textit{forms} of gestures
in the same way, i.e., all gestures have the same left-right Bakis topology for their
nucleus phases. The fourth column is the result from using a left-right Bakis
topology for path gestures and a single state topology for pose gestures. To
ensure a fair comparison, all hidden states have 3 mixtures of Gaussians. As Table~\ref{tab:result}
shows, having different HMM topologies for the two \textit{forms} of gestures
significantly increases the recall and $F_1$ score for pose gestures.

\begin{table}[tbh]
\centering
\begin{tabular}{|l|l|p{3cm}|p{3cm}|}
\hline
& & \textbf{Same topology for two forms of gestures} & \textbf{Different
topologies for two forms of gestures} \\
\hline
\multirow{4}{4cm}{Path \& discrete flow gestures} 
& Precision & \textbf{0.84 (0.16)} & 0.82 (0.16) \\
\cline{2-4}
& Recall & 0.87 (0.17) & 0.87 (0.18)\\
\cline{2-4}
& $F_1$ & \textbf{0.85 (0.16)} &  0.84 (0.16)\\
\cline{2-4}
& Responsiveness (s) & 0.6 (0.3) & 0.6 (0.3) \\
\hline
\multirow{4}{4.5cm}{Pose \& continuous flow gestures}
& Precision & \textbf{0.65 (0.14)} & 0.63 (0.11) \\
\cline{2-4}
& Recall & 0.41 (0.15) & \textbf{0.65 (0.09)} \\
\cline{2-4}
& $F_1$ & 0.50(0.14) & \textbf{0.64 (0.10)} \\
\hline
\textbf{Average} & $F_1$ & 0.675 (0.150) & \textbf{0.740 (0.130)} \\
\hline
\end{tabular}
\caption{Results from using different topologies. The numbers in parentheses are
standard deviations. The results are based on using 3 mixtures of Gaussians
for all hidden states, and lag time
$l = 8$ frames.}
\label{tab:result}
\end{table}

For gestures with arbitrary movement (e.g., pose gestures), it is difficult to
use embedded training to accurately align pre-stroke, nucleus and post-stroke
phases, because training sequences can have very different lengths, and so do
the testing sequences.
Figure~\ref{fig:palm-hidden} shows the estimates of
the most likely hidden states for a Palm Up gesture sequence when the same
left-right topology are used for both \textit{forms} of gestures is used.
The main (center) part of the gesture, which should be the nucleus of the gesture, is identified
as the post-stroke. This is why it is important to have different topologies and
different training strategies for the two forms of gestures.

\begin{figure}[tbh]
\centering
\includegraphics[width=0.3\linewidth]{figures/palm_hidden_label.png}
\caption{Estimated hidden states for a Palm Up gesture using the
left-right model the same as path gestures. Different colors correspond to
different hidden states.}
\label{fig:palm-hidden}
\end{figure}

\section{Effect of Different Numbers of Mixtures}
The previous section shows that using different topologies for path and pose
gestures give better
recognition performance. So using this model, I investigate the effects
of the number of mixtures ($k$) in the MoG emission probabilities.

First, I set $k$ be the same for both forms of gestures.
Fig.~\ref{fig:mixtures} shows that the $F_1$ score increases as the number of
mixtures increases until $k=3$.
After that, we start to see the effect of overfitting.

\begin{figure}[tbh]
\centering
\includegraphics[trim=10mm 5mm 10mm 15mm,
clip, width=\columnwidth]{figures/f1_nM.png}
\caption{F1 scores versus number of mixtures.}
\label{fig:mixtures}
\end{figure}

I then experimented with using different $k$'s for path and pose gestures. For
path gestures, I
set $k^\text{path} = 3$, and use a different number of
mixtures, $k^{\text{pose}}_g\in \{3\ldots6\}$, for each pose gesture $g$. Each
$k^{\text{pose}}_g$ is chosen according to the Bayesian Information Criterion
(see Section~\ref{sec:pose-gesture}).
Using this method, I am able to improve precision, recall and $F_1$ scores for
both forms of the gestures even further (see
Table~\ref{tab:different-mixtures}).

\begin{table}[tbh]
\centering
\begin{tabular}{|p{4.5cm}|l|p{4cm}|}
\hline
& & \textbf{Use different topologies and numbers of mixtures} \\
\hline
\multirow{4}{4cm}{Path \& discrete flow gestures} 
& Precision & \textbf{0.94 (0.05)} \\
\cline{2-3}
& Recall    & \textbf{0.93 (0.08)} \\
\cline{2-3}
& $F_1$ & \textbf{0.93 (0.06)} \\
\cline{2-3}
& Responsiveness (s) & 0.6 (0.2)  \\
\hline
\multirow{3}{4cm}{Pose \& continuous flow gestures}
& Precision & \textbf{0.68 (0.10)} \\
\cline{2-3}
& Recall & \textbf{0.69 (0.08)} \\
\cline{2-3}
& $F_1$ & \textbf{0.68 (0.09)}  \\
\hline
\textbf{Average} & $F_1$ & \textbf{0.805 (0.075)}\\
\hline
\end{tabular}
\caption{Results from using different numbers of mixtures of Gaussians
for the emission probabilities ($l = 8$ frames).}
\label{tab:different-mixtures}
\end{table}

%Compare user dependent vs user independent result

\section{Effect of Different Lag Times}
Figure~\ref{fig:lag} shows how the $F_1$ scores change with respect to the lag
time ($l$) in fixed-lag smoothing. The performance increases as $l$ increases, and
plateaus at $l=8$ frames which is about 0.3s at 30 FPS. This shows that more
evidence does help to improve the estimates until a limit, and we do not need to
sacrifice too much delay to reach the limit. Our result also shows that the
responsiveness score (RS, defined in Equation~\ref{eqn:rs}) stays around 0.6--0.7
seconds as $l$ is increased.

\begin{figure}[!tbh]
\centering
\includegraphics[trim=0 5mm 0 15mm, clip,
width=\columnwidth]{figures/f1_lag.png}
\caption{$F_1$ score versus lag time $l$.}
\label{fig:lag}
\end{figure}

\section{Training Time}
The HMM-based unified framework is fast to train. The average
computation time for training the model for one user is about 5s with 7 gestures and 6 training examples per
gesture. 

Because of the fast training process, the system is easily extensible. New
gestures can be added by recording 3-6 repetitions of the gesture using the Kinect; the system will train an HMM model for the gesture
and add it to the existing combined HMM. This process, including the recording
time, takes only a few minutes.

\section{User Independent Evaluation}
Our survey (Section~\ref{sec:preferences}) indicated that users
generally prefer a system to have a predefined gesture set and then add or
change gestures later according to their own preferences. This means that having
a user independent base model will be useful as well. If the system has user
provided training examples for certain gestures, it uses the
user dependent models for those gestures; otherwise it backs off to the base
model. This adaptation strategy is similar to~\cite{yin13-making}.

To evaluate the user independent model, I did a 10-fold cross-validation where
in each fold, the data from one user's last 2 sessions are used for testing and
the data from the remaining 9 users' first 2 sessions are used for training.
Table~\ref{tab:user-independent} shows the average, best and worst results among
the users. We can see that there is large variation among the users. Users
who want to do the gestures differently need to train their user
dependent models. For example, the results based on user dependent model for user PID-02
is much better than the user independent model.

\begin{table}[tbh]
\centering
\begin{tabular}{|p{3cm}|l|l|p{1.9cm}|p{1.9cm}||p{2cm}|}
\hline
& & \textbf{Average} & \textbf{Best (PID-01)} & \textbf{Worst (PID-02)} &
\textbf{User dep. (PID-02)} \\
\hline
\multirow{4}{3cm}{Path \& discrete flow gestures} 
& Precision & 0.72 (0.18) & 1.00 & 0.35 & 0.96 \\
\cline{2-6}
& Recall    & 0.72 (0.16) & 1.00 & 0.72 & 1.00\\
\cline{2-6}
& $F_1$ & 0.70 (0.16) & 1.00 & 0.47 & 0.98 \\
\cline{2-6}
& Responsiveness (s) & 0.6 (0.3)  & 0.6 & 0.4 &  0.4\\
\hline
\multirow{3}{3cm}{Pose \& continuous flow gestures}
& Precision & 0.59 (0.12) & 0.79 & 0.49 & 0.67\\
\cline{2-6}
& Recall & 0.56 (0.16) & 0.77 & 0.21 & 0.67\\
\cline{2-6}
& $F_1$ & 0.56 (0.13)  & 0.78 & 0.29 & 0.67\\
\hline
\textbf{Average} & $F_1$ & 0.63 (0.15) & 0.89 & 0.38 & 0.83\\
\hline
\end{tabular}
\caption{User independent model 10-fold cross validation results ($l = 8$
frames). The last column is the user dependent result for user PID-02 for
comparison.}
\label{tab:user-independent}
\end{table}

\section{Discussion}
Our overall best performance for the YANG dataset is reported in
Table~\ref{tab:different-mixtures}. The performance for the pose and
continuous flow gestures is relatively low compared to that of the path and
discrete flow gestures. Most
confusions are among the 3 pose gestures, i.e., Point, Palm Up, Grab (see
Figure~\ref{fig:recog-result} for example).
There are big variations in the recognition results for the pose gestures among users: the highest $F_1$ score for pose gestures is 0.81
and the lowest is 0.58. For the ones with low $F_1$ scores, confusion occurs
when there is significant motion blur (see Figure~\ref{fig:point_grab}). These
correspond to the users who move relatively fast when doing the pose gestures.
This suggests that with the frame rate of the Kinect sensor used (30Hz for
the Kinect version One), users may need to perform pose gestures relatively
slowly in order to get better recognition results, or a better (higher rate)
sensor is needed.

\begin{figure}[tbh]
\centering
\includegraphics[width=\linewidth]{figures/point_posture_full_image.png}
\caption{This frame is mistakenly classified as Grab while the true gesture
is Point. Motion blur is significant.}
\label{fig:point_grab}
\end{figure}