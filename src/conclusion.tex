\chapter{Conclusion}
We designed our system from natural human computer interaction perspective.
Our evaluation shows promising results for our unified probabilistic framework
for real-time gesture recognition that handles two forms of gestures. Using the
framework, we developed a gesture controlled presentation application similar to
the one described at the beginning of this paper.

The performance for gestures with distinct hand poses is lower than that for
gestures with distinct paths. This may be due to limit in
both the pixel and the depth resolutions of the Kinect sensor. It would be interesting to test
the new version of the Kinect sensor which uses a time-of-flight depth sensor
and is reported to have a higher resolution. We are also going to to explore
other feature descriptors and encoding methods to see whether we can
improve the result for gestures with distinct hand poses.

Using embedded training and hidden state information, we can effectively
detect gesture phases, allowing the system to respond more promptly. On average,
for discrete flow gestures, the system responds 0.6s before the hand comes to
rest. 

We collected a new dataset that includes two forms of gestures, a
combination currently lacking in the community, and plan to make it
public. We have also proposed a hybrid evaluation metric
that is more relevant to real-time interaction and different types of gestures.