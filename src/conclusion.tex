\begin{savequote}
Language is a part of social behavior. What is the mechanism whereby the social
process goes on? It is the mechanism of gesture\ldots
\qauthor{George Herbert Mead, \textit{Mind, self, and society}}
\end{savequote}
\chapter{Conclusion}
I developed a real-time continuous gesture recognition system from a natural
human computer interaction perspective.
The evaluation shows promising results for the unified probabilistic framework
that handles two forms of gestures. Using the
framework, I developed a gesture controlled presentation application similar to
the one described at the beginning of this paper. All the code is open-source
and is available
online\footnote{\url{http://groups.csail.mit.edu/mug/projects/gesture_kinect/index.html\#code}}.
My previous publications related to this thesis are~\cite{yin12, yin10, yin13,
yin13-making}.


Using embedded training and hidden state information, we can effectively
detect gesture phases, allowing the system to respond more promptly. On average,
for discrete flow gestures, the system responds 0.6s before the hand comes to
rest. 

We collected a new dataset that includes two forms of gestures, a
combination currently lacking in the community, and plan to make it
public. We have also proposed a hybrid evaluation metric
that is more relevant to real-time interaction and different types of gestures.

\section{Limitations}
\section{Future Work}
The performance for gestures with distinct hand poses is lower than that for
gestures with distinct paths. This may be due to limit in
both the pixel and the depth resolutions of the Kinect sensor. It would be interesting to test
the new version of the Kinect sensor which uses a time-of-flight depth sensor
and is reported to have a higher resolution. We are also going to to explore
other feature descriptors and encoding methods to see whether we can
improve the result for gestures with distinct hand poses.

Combine Leap Motion and smartwatches.

Better base model using mixture of HMMs.
User adaptation: weighted combination