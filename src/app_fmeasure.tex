\chapter{Review of F-measure}\label{app:fmeasure}
In statistical analysis of binary classification, the F-measure is a measure of
test's accuracy, combining both the precision and the recall of the test.
Precision is the fraction of correct results from all the returned results
(i.e., the number of correct results divided by the number of all returned
results), and recall is the fraction of correct results from all the results
that should be returned (i.e., the number of correct results divided by the
number of results that should have been returned)~\cite{f1score14}. Note that
both precision and recall have the same numerator.

The general formula for F-measure is
\begin{align*}
F_\beta &= (1 + \beta^2)\cdot\frac{\text{precision}\cdot
\text{recall}}{(\beta^2\cdot\text{precision}) + \text{recall}}\\
&= \frac{1 + \beta^2}{\frac{1}{\text{precision}}+\frac{\beta^2}{\text{recall}}}
\end{align*}
which is the weighted harmonic mean of precision and recall. The $F_1$ score
($\beta=1$) is the most commonly used one where precision and recall have
equal weight:
\begin{align*}
F_1 = 2\cdot\frac{\text{precision}\cdot
\text{recall}}{\text{precision} + \text{recall}}
\end{align*}
Two other commonly used F measures are the $F_2$ measure, which weighs recall
higher than precision, and the $F_{0.5}$ measure, which puts more emphasis on
precision than recall.

