\chapter{Hybrid Performance Metrics}
Both frame \cite{song12} and event-based \cite{guyon13} metrics has been
used for evaluating gesture recognition systems. A \textit{frame} is a
fixed-length, fixed-rate unit of time. It is often the smallest unit of measure
defined by the system \cite{ward11} and in such cases approximates continuous
time.
For example, in our case, a frame is a data frame consisting of a RGB data frame, a depth data frame and a skeleton data from the
sensor at 30 FPS. An \textit{event} is a variable duration sequence of frames
within a continuous time-series.  It has a start time and a stop time.

Each of these metrics alone may not be adequate for evaluating a real-time
gesture recognition system handling different types of gestures. Frame-based
evaluation is less relevant for gestures requiring discrete responses. For the ChaLearn Gesture Challenge 2012, Guyon et al.
\cite{guyon13} use the Levenshtein distance between the ordered list of
recognized events and the ground truth events. However, such event-based metrics
that ignore timing of recognition are inappropriate for real-time
applications where responsiveness of the system matters.

Ruffieux et al. combined a time-based metric with an event-based metirc
\cite{Ruffieux2013}. Ward et al. \cite{ward11} proposed a comprehensive
scheme of combining both frame and event scoring.
We believe that all three types of information -- frames, events and
timings -- are relevant for a real-time system that responds to different types
of gestures. Hence, we developed a hybrid performance metric.

\section{Metrics for Discrete Flow Gestures}
For discrete flow gestures, the system responds at the end of the nucleus phase,
so the evaluation should be event-based. Let $T_{{\text{gt\_start\_pre}}}$ be
the ground truth start time of the pre-stroke phase and
$T_{{\text{gt\_stop\_post}}}$ be the ground truth stop time of the post-stroke
phase.
A recognized event is considered a true positive (TP) if the time of response ($T_{\text{response}}$) 
occurs between $T_{{\text{gt\_start\_pre}}}$ and $T_{{\text{gt\_stop\_post}}} +
0.5\times(T_{{\text{gt\_stop\_post}}} - T_{\text{gt\_start\_pre}})$. We allow
some margin for error because there can be small ground truth timing errors.
Once a TP event is detected, the corresponding ground truth event is not
considered for further matching, so that multiple responses for the same gesture
will be penalized. We then can compute event-based precision, recall and F1 score.
\begin{align}
\text{precision} &=\frac{\text{\# TP events}}{\text{\# recognized events}} \\
\text{recall} &=\frac{\text{\# TP events}}{\text{\# ground truth events}} \\
F_1 &= 2\cdot \frac{\text{precision} \cdot \text{recall}}{\text{precision} +
\text{recall}}
\end{align}

For discrete flow gestures, we also define a Responsiveness Score (RS) as the
time difference in seconds between the moment when the system responds and the moment when the hand goes to a rest
position or changes gesture. Let $N_{TP}$ be the number of true positives, then
\begin{align}
RS = \frac{\sum_{i = 1}^{N_{TP}}T_{{\text{gt\_stop\_post}}} -
T_{\text{response}}}{N_{TP}}
\end{align}
A positive score means the responses are before the end of the post-strokes,
hence higher scores are better.

For continuous flow gestures, the system responds frame by frame,
so it is more appropriate to use frame-based evaluation. For all the frames that are
recognized as continuous gestures, we can compute the number of TPs by
comparing with the corresponding frame in the ground truth. Then, we can
compute frame-based precision, recall and F1 score for all the frames
corresponding to continuous flow gestures.

The average of the two F1 scores can give an overall indication of the
performance of the system.
