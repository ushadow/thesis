\chapter{Notation and Abbreviations}\label{app:notation}
We adopt the standard convention that random variables are denoted as captial
letters, and instantiations of random variables (values) are denoted as
lower-case letters. We use underlines for vector-valued quantities to
distinguish them from scalar-valued ones. So $x$ refers to a scalar, while
$\underline{x}$ refers to a vector. We use caligraphic letters to denote sets.

\begin{table}[tbh]
\centering
\begin{tabular}{|l|l|}
\hline
\thead{Symbol}  & \thead{Meaning} \\
\hline
$S_t$     & Hidden state variable at time $t$ \\
\hline
$X_t$     & Observation (output) at time $t$ \\
\hline
$T$       & Length of sequence \\
\hline
$\underline{x}_{1:T}$ & Sequence of observation \\
\hline
\end{tabular}
\caption{Notation for general state-space models.}
\end{table}

\begin{table}[tbh]
\centering
\begin{tabular}{|l|l|}
\hline
\thead{Abbreviation} & \thead{Meaning} \\
\hline
ATSR & Accurate temporal segmentation rate \\
\hline
CF & Continuous flow \\
\hline
CRF & Conditional random fields \\
\hline
DF & Discrete flow \\
\hline
FPS & Frame per second \\
\hline
IMU & Inertia measurement unit \\
\hline
HCI & Human computer interaction \\
\hline
HMM & Hidden Markov model\\
\hline
HOG & Histogram of oriented gradients \\
\hline
PCA & Principal component analysis \\
\hline
SVM & Support Vector Machine \\
\hline
TP & True positive \\
\hline
TPR & True positive rate \\
\hline
\end{tabular}
\caption{List of abbreviations}
\end{table}
